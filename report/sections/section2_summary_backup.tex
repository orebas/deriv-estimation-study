\section{Summary of Key Findings}
\label{sec:summary}

Before detailing our experimental journey, we present the primary conclusions of this study. After a comprehensive evaluation across a range of methods, three distinct dynamical systems, and a sweep of noise levels, a clear verdict emerges. Performance was analyzed in two regimes: a "low-noise" regime ($\le 0.1\%$ noise) where precision is paramount, and a "high-noise" regime (1-2\% noise) where robustness is key.

The table below summarizes the performance of contender methods that demonstrated full data coverage for derivative orders 1 through 5. Methods are sorted by their overall average rank, giving equal weight to performance in both low- and high-noise regimes. Averages are calculated over orders 1-5 (excluding order 0 function approximation, focusing on actual derivative estimation).
% should we exclude 0 in the task?

% AUTO-GENERATED by gemini-analysis/generate_exploratory_tables.py
% To regenerate: ./scripts/04_generate_tables.sh
% Alternative versions (orders 3, 7) available as tab_summary_order{3,7}.tex
\input{../build/tables/publication/tab_summary_order5.tex}

\textbf{Our principal findings are as follows:}

\begin{enumerate}
    \item \textbf{Gaussian Process Regression (GPR) is the most robust and accurate method overall.} The Julia GPR implementation (\texttt{GP-TaylorAD-Julia}) and the improved Python variants (\texttt{GP-RBF-*}) are the clear winners, consistently occupying the top ranks in both low and high-noise regimes.
    \item \textbf{The optimal method depends on the noise level and derivative order.} While GPR is the best all-arounder, splines like \texttt{Spline-Dierckx-5} offer excellent precision in low-noise environments, making them a top choice for cleaner data, particularly at modest derivative orders. In the high-noise regime, the filtering-based \texttt{Savitzky-Golay} provides a computationally cheap and highly effective alternative, ranking solidly in the top half of contenders.
    \item \textbf{Theoretical limits matter.} Many common low-degree spline methods are, by definition, incapable of representing high-order derivatives, limiting their applicability.
    \item \textbf{Dedicated packages offer convenient and robust options, but need a differentiation backend.}   Purpose built libraries are convenient, but our strongest results come from extending them with either analytic or auto-differentiated derivatives of smoothed data.  For this to work best, the library should ideally produce very smooth models, and either expose some of the model internals or be amenable to AD.

\end{enumerate}

The subsequent sections of this paper will detail the experimental journey and analysis that led to these conclusions, providing a narrative account of our investigation and offering a practical framework for method selection.

\subsection{Visual Confirmation of Findings}

The quantitative results in the summary table are powerfully illustrated by a few key visualizations.

\begin{figure}[htbp]
\centering
\includegraphics[width=0.9\textwidth]{high_noise_fit_comparison.png}
\caption{\textbf{High-Noise Performance (2\% noise, 4th derivative).} Three-panel comparison showing: (top) noisy input data as black points against gray ground truth, emphasizing the challenging signal-to-noise environment; (middle) derivative estimates from four methods representing different algorithmic approaches; (bottom) estimation errors revealing where each method succeeds or fails. The figure shows \texttt{GP-TaylorAD-Julia}, \texttt{SavitzkyGolay-Fixed}, \texttt{Spline-Dierckx-5}, and \texttt{Fourier-GCV}. The error panel reveals that all methods struggle at the trajectory boundaries.}
\label{fig:high_noise_comp}
\end{figure}

\begin{figure}[htbp]
\centering
\includegraphics[width=0.9\textwidth]{low_noise_fit_comparison.png}
\caption{\textbf{Low-Noise Precision (1e-6 noise, 5th derivative).} Three-panel comparison in a low-noise regime: (top) noisy data points (black) are nearly imperceptible against gray ground truth; (middle) 5th-order derivative estimates from the top three methods; (bottom) estimation errors showing clear performance hierarchy. \texttt{GP-TaylorAD-Julia} dominates with RMS error of 272k, while \texttt{Spline-Dierckx-5} (3.1M) and \texttt{Fourier-GCV} (5.8M) show substantially larger errors. Even in this clean environment, high-order differentiation remains extremely challenging, with \texttt{GP-TaylorAD-Julia} achieving more than 10× better accuracy than competing methods.}
\label{fig:low_noise_comp}
\end{figure}

\begin{figure}[htbp]
\centering
\includegraphics[width=0.9\textwidth]{top_methods_heatmap.png}
\caption{\textbf{Performance Degradation with Increasing Derivative Order.} This heatmap shows the mean nRMSE for top methods at a 1\% noise level. The vertical axis is sorted by average performance across all orders. This visualization clearly shows that \texttt{GP-TaylorAD-Julia} maintains low error across all derivative orders, while other methods like \texttt{Spline-Dierckx-5} and \texttt{Spline-GSS} are highly accurate for low orders but degrade significantly as the order increases.}
\label{fig:heatmap}
\end{figure}

% Computational efficiency considerations are discussed in Section~\ref{sec:efficiency} rather than as a primary finding, due to implementation-dependent variability in timing measurements.