\section{Conclusion}
\label{sec:conclusion}

This comprehensive study evaluated a wide array of numerical methods for the estimation of high-order derivatives from noisy data. After a detailed investigation that included a multi-stage filtering of methods and a deep dive into implementation details, our findings are clear and decisive.

\subsection{Summary of Key Findings}

\begin{itemize}
    \item \textbf{Gaussian Process Regression (GPR) is the most robust and accurate method overall.} GPR methods consistently occupy the top ranks in both low- and high-noise regimes, making them the most reliable choice for general-purpose derivative estimation.
    \item \textbf{The optimal method depends on the use case.} While GPR is the best all-arounder, splines like \texttt{Dierckx-5} offer excellent precision for low-noise data, while spectral methods like \texttt{Fourier-Continuation} provide a compelling balance of speed and accuracy. For speed-critical applications, \texttt{Savitzky-Golay} is a robust and effective baseline.
    \item \textbf{Derivative order is the dominant difficulty factor.} Performance degrades systematically with increasing order across all methods. The problem becomes significantly more challenging beyond order 3, and only a handful of methods produce usable results at orders 6 or 7.
    \item \textbf{Implementation quality is a critical method characteristic.} Our study found significant performance differences between different software packages implementing the same underlying algorithm, highlighting that practitioners must consider the quality of a specific implementation, not just the theoretical method.
\end{itemize}

\textbf{The primary recommendation of this work is that for practitioners who require accurate high-order derivatives from real-world, noisy signals, Gaussian Process Regression is the most reliable and effective starting point.}

\subsection{Practitioner's Guide to Method Selection}
\label{sec:practitioners_guide}

The selection of an appropriate numerical differentiation method depends critically on your specific application constraints. Based on our comprehensive analysis, we provide the following practical guidance:

\textbf{Decision Framework:}

\begin{enumerate}
    \item \textbf{If real-time performance is essential (sub-millisecond):}
    \begin{itemize}
        \item Use \texttt{Savitzky-Golay} for moderate noise levels (up to 1\%)
        \item Consider basic FFT approaches for periodic signals
        \item Accept that accuracy will be limited for orders $> 2$
        \item Implementation tip: Pre-compute filter coefficients when possible
    \end{itemize}

    \item \textbf{If you need balanced accuracy and speed (1-100 ms):}
    \begin{itemize}
        \item Use \texttt{Fourier-GCV} or \texttt{Fourier-Continuation} for smooth, periodic data
        \item Use \texttt{Dierckx-5} (degree-5 splines) as a robust generalist
        \item These methods handle orders 0-5 effectively
        \item Implementation tip: Fourier methods benefit from power-of-2 data lengths
    \end{itemize}

    \item \textbf{If maximum accuracy is paramount (100+ ms acceptable):}
    \begin{itemize}
        \item Use \texttt{GP-Julia-AD} for the most robust performance across all conditions
        \item Essential for derivative orders $\geq 5$
        \item Provides best noise robustness (handles 2\% noise effectively)
        \item Implementation tip: Ensure Taylor-mode AD is available for orders $> 3$
    \end{itemize}
\end{enumerate}

\textbf{Additional Considerations:}

\begin{itemize}
    \item \textbf{Dataset size:} For $N > 1000$ points, be aware that GP methods scale as $O(N^3)$. Consider spectral methods ($O(N \log N)$) or local filters ($O(N)$) for large datasets unless sparse GP approximations are available.

    \item \textbf{Noise characteristics:} If noise level is unknown, GP methods with built-in noise estimation provide the most robust results. For known low noise ($< 0.01\%$), spline methods offer excellent precision.

    \item \textbf{Derivative order:} For orders 0-2, most methods perform adequately. For orders 3-5, use spectral or GP methods. For orders 6-7, only GP methods with Taylor-mode AD remain viable.

    \item \textbf{Implementation quality:} When possible, use established libraries rather than reimplementing. Our results show significant performance variations between implementations of the same algorithm.
\end{itemize}

\textbf{Quick Reference Table:}

\begin{center}
\small
\begin{tabular}{llll}
\hline
\textbf{Scenario} & \textbf{Recommended Method} & \textbf{Speed} & \textbf{Max Order} \\
\hline
Real-time control & Savitzky-Golay & $<$1 ms & 2-3 \\
Signal processing & Fourier-GCV & 1-10 ms & 5 \\
Scientific analysis & Dierckx-5 & 10-50 ms & 5 \\
High-precision needs & GP-Julia-AD & 100-500 ms & 7 \\
\hline
\end{tabular}
\end{center}

\subsection{Future Work}

This benchmark, while comprehensive, is not exhaustive. Several avenues for future research are immediately apparent:

\begin{enumerate}
    \item \textbf{Testing on Diverse Signals:} Our study used ODEs that produce smooth, analytic signals. Future work should include testing on more challenging signals, such as those with discontinuities, sharp peaks, or chaotic behavior.
    \item \textbf{Evaluating Alternative Noise Models:} The real world is not always Gaussian. A valuable extension would be to evaluate method performance under different noise models, such as multiplicative, Poisson, or heavy-tailed noise.
    \item \textbf{Larger-Scale Problems:} Our study was limited to a modest number of data points. Investigating how method performance, particularly computational cost, scales to much larger datasets ($N > 1000$) would be of great practical interest.
\end{enumerate}

\subsection{Broader Implications: The Case for a Composable, Differentiable Ecosystem}

Our findings also underscore a broader trend in scientific computing: the immense value of composable and differentiable software packages. The "Approximant-AD" framework is only possible when libraries for data modeling (e.g., Gaussian Processes) can seamlessly pass their results to libraries for automatic differentiation.

While not all numerical packages are readily differentiable out-of-the-box, our experience suggests that many can be adapted with modest effort. We encourage researchers and developers to prioritize differentiability in their own software and to contribute upstream to make foundational libraries in the ecosystem compatible with AD frameworks. Such efforts create a virtuous cycle, unlocking powerful new hybrid methodologies that benefit the entire scientific community, far beyond the immediate application of derivative estimation.
